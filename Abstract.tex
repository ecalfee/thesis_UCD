\begin{center}
   \textbf{Impact of Introgression on Adaptation and Range Expansions} \\
\end{center}

\section*{Abstract}

	Gene flow between diverged populations is often deleterious (e.g. due to genetic incompatibilities or local adaptation),
	but introgression can also be a source of rapid adaptation, especially to novel niches.
	My thesis focuses on the evolutionary outcomes of these opposing selection forces on introgressed ancestry,
	the repeatability of these outcomes across admixed populations,
	and the consequences for species niches.
	I have developed novel population genetics methods to detect selection in admixed populations,
	and applied these methods to disentangle how demography and selection have shaped the evolution
	and range expansions of two very different species: \textit{scutellata}-European hybrid honey bees and highland maize.
	
	For my first dissertation chapter, I conducted a cross-continental comparison of the outcomes of admixture and selection in \textit{scutellata}-European honey bees.
	\textit{Scutellata} honey bees from South Africa were introduced to the Americas in the 1950s, but soon escaped, and through interbreeding with European honey bees,
	formed a highly successful and invasive hybrid population that spread at $\sim$300km/year across the Americas.
	This is a great system to study the adaptive potential and the limitations of admixture in facilitating rapid range expansions,
	with natural replication in North and South America.
	For this research, I collected and sequenced 300+ bees from two nearly 1000 km transects,
	one in California and one in Argentina, to compare ancestry patterns across their genomes.
	I found evidence of convergent selection favoring African \textit{scutellata} honey bee ancestry at a number of loci in the genome in North and South America.
	These loci are strong candidates for contributing to the high fitness and success of \textit{scutellata}-European hybrid honey bees.
	Because these bees are highly defensive, their continued spread is an agricultural and public health concern.
	I found parallel clines in genomewide ancestry between continents at similar latitudes,
	despite much larger dispersal distances to reach California from the origin of the invasion,
	evidence that many loci across the genome are currently preventing spread to temperate zones.
	
	For my second dissertation chapter, I analyzed the outcome of introgression between maize and its wild highland ‘teosinte' relative, \textit{mexicana},
	which may have facilitated maize’s range expansion from the valleys where it was domesticated up to 3000m in the mountains of Mexico.
	Using a novel method that accounts for background patterns of ancestry variance and covariance between populations
	(e.g. due to gene flow or shared drift post-admixture), I found strong evidence for adaptive introgression from \textit{mexicana} into maize,
	especially among the highest elevation populations, consistent with introgression facilitating maize’s colonization of the highlands.
	I also found loci (including a newly identified inversion) where selection maintains steep ancestry clines across elevation.
	I demonstrated evidence of selection against introgression,
	removing \textit{mexicana} ancestry from near domestication genes and lower recombination regions of the genome (due to linked selection).
	One surprising finding is that despite observations of hybrids in the field,
	and opportunities for gene flow from locally adapted \textit{mexicana} that grows side-by-side with contemporary maize,
	I found little evidence for recent locally sourced haplotypes genomewide or at loci with high local introgression.
	Rather, the majority of introgression is from over 1000 generations ago,
	and has subsequently diverged within the maize background and been sorted by selection along an elevational cline and within individual populations.
	This work has broader impacts for understanding the longer term effects of introgression on range expansions and aiding in the discovery of key loci associated with high-elevation adaptations,
	which may be crucial for future breeding of maize, a global staple, under climate change.
	
	Overall, this thesis adds to our knowledge of the role of introgression in range and niche expansions,
	and provides in-depth genomic analyses of selection and admixture in two agriculturally-important species.
